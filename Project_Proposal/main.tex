\documentclass[12pt]{article}
\usepackage[acronym]{glossaries}

\usepackage{supertabular} %Si jamais des tableaux vont sur plus qu'une page

\usepackage{biblatex}
\usepackage{calligra}
\usepackage{graphicx,wrapfig,lipsum}
\usepackage[T1]{fontenc}  %Pour faire des ë,ö,û,...
\usepackage{icomma}       %Pour écrire 1,8 au lieu de 1, 8 (espace après la virgule en mode math)
\usepackage{physics}
\usepackage{siunitx}
\usepackage{array}     %Pour pouvoir faire des matrices et tableaux
\usepackage{color}     %Pour écrire du texte en couleur
\usepackage{amsmath,mathtools}   %Pour écrire des équations
\usepackage{amssymb,amsfonts}   %Pour loader des variables, lettres, flèches, signes...
\usepackage{esint}     %Pour faire des intégrales fermées et doubles
\usepackage{multirow}  %Pour fusionner des cases dans un tableau
\usepackage{float}     %Pour les figures flottantes et placer les figures dans le texte
\usepackage{graphicx}  %Pour intégrer des graphiques, images et photos..
\usepackage{tikz}   %dessiner toute sorte de belles choses.. Pratiquement sans fin ;)
\usepackage[left=2.5cm,right=2.5cm,top=3cm,bottom=3cm]{geometry} %Pour controler les marges
\usepackage{adjustbox}
\usepackage{hyperref}  %Pour mettre des liens de références dans le texte

\usepackage{caption}   %Pour pouvoir mettre des légendes aux figures «captions»
\usepackage[bottom]{footmisc}
\usepackage{gensymb} 
\usepackage{tabulary}  
\usepackage{fancyhdr}  
\usepackage{siunitx}   %Pour écrire des unités du système international
\usepackage{textcomp}
\usepackage{amsmath}
\newcommand{\HRule}{\rule{\linewidth}{0.5mm}}
\counterwithin*{equation}{section}
\usepackage{hyperref}
\usepackage{tcolorbox}
\usepackage{subfigure}
\usepackage{hhline}

\usepackage{pythonhighlight}
\usepackage{cancel}
% R CURSIF

\usepackage{calligra}
\newcommand{\angstrom}{\mbox{\normalfont\AA}}
\DeclareMathAlphabet{\mathcalligra}{T1}{calligra}{m}{n}
\DeclareFontShape{T1}{calligra}{m}{n}{<->s*[2.2]callig15}


%%%%%


\makeatletter
\newcommand{\thickhline}{%
    \noalign {\ifnum 0=`}\fi \hrule height 1pt
    \futurelet \reserved@a \@xhline
}
\newcolumntype{"}{@{\hskip\tabcolsep\vrule width 1pt\hskip\tabcolsep}}
\makeatother
\counterwithin*{equation}{section}
\usepackage{parskip}  
\usepackage{braket}

\setlength{\parindent}{0pt} 
\setlength{\parskip}{15pt}  
\renewcommand{\baselinestretch}{1.2}
\newcommand*\rfrac[2]{{}^{#1}\!/_{#2}}


\usepackage{fancyhdr}
\setlength{\headheight}{35.1pt}
\pagestyle{fancy}
\rhead{\footnotesize 2024}
\lhead{\footnotesize{Project Proposal}}
\renewcommand\headrulewidth{0.7pt}
\usepackage[T1]{fontenc}
\usepackage[english]{babel}
\usepackage{listings}
\usepackage[utf8]{inputenc}
\usepackage{biblatex} %Imports biblatex package
\usepackage{enumitem}
\newlist{todolist}{itemize}{2}
\setlist[todolist]{label=$\square$}
\usepackage{pifont}
\newcommand{\cmark}{\ding{51}}%
\newcommand{\xmark}{\ding{55}}%
\newcommand{\done}{\rlap{$\square$}{\raisebox{2pt}{\large\hspace{1pt}\cmark}}%
\hspace{-2.5pt}}
\newcommand{\wontfix}{\rlap{$\square$}{\large\hspace{1pt}\xmark}}

\addbibresource{Biblio.bib} %Import the bibliography file\usepackage{csquotes}

\newacronym{amt}{AMT}{Automatic Music Transcription}
\newacronym{mir}{MIR}{Music Information Retrieval}
\newacronym{nn}{NN}{Neural Network}
%\newacronym{}{}
\begin{document}
\begin{titlepage}
\begin{center}
\includegraphics[scale=0.35]{figs/ul_logo.pdf} 
\line(1,0){450}\\
[2mm]
\begin{large}
\textbf{Project Proposal} \\ 
\end{large}
\line(1,0){450}\\
[1.5cm]
Realized by\\
Antoine Veillette\\
Kibalou Betchaleel Banakinao\\
Simon Ferland\\
[3cm]
As part of the course\\
Machine Learning for Signal Processing\\ 
[2.5cm]
Work presented to\\
M. Cem Subakan\\
[3cm]
Department of computer science and software engineering\\
Faculty of science and engineering\\
Laval university\\

\today
\end{center}
\end{titlepage}
\begin{abstract}
\acrfull{amt}  is a fundamental problem in 
\acrfull{mir}. In \cite{Zhang2020}, it was recommended that the pseudo bispectrum analysis model could be improved by implementing post-processing using neural network. The following project goal is to implement such post processing to improve the result obtained by \cite{Zhang2020}.

\end{abstract}
\section{Introduction}

\section{Litterature review}
in \cite{Liu2020}, they used \acrlong{nn} in order to do \acrlong{amt} 
\section{Input-Output-dataset}
The dataset used will be the MAESTRO V3 dataset\cite{hawthorne2018enabling}
\section{Performance metric}
the performance metrics will be evaluated using the Mir\_eval librairy. The main metric will be the $f$ measure with $\beta = 1$.
\section{Computationnal ressources}
We will use Valeria to train our model.
\section*{Checklist}

\begin{todolist}
    \item[\done] \textbf{Realism:} The project will be completed within a reasonable time frame, using available resources like Google Colab.
    
    \item[\done] \textbf{Well-defined dataset:} We will use the Maestro v3 dataset to label our pseudo 2D spectrums with the right note labels.
    
    \item[\done] \textbf{Clear input-output definition:} The input will be song in wav format and the output will be a pianoroll of the song.
    
    \item[\done] \textbf{Non-trivial:} The project explores a novel approach to AMT using pseudo 2D spectrum as the input of our neural nets as proposed in \cite{Zhang2020}. 
    
    \item[\done] \textbf{Machine learning/signal processing focus:} The project incorporates machine learning techniques, specifically [ML/SP technique], ensuring concrete predictions.
    
    \item[\done] \textbf{Literature grounding:} A literature review is included, situating the project within existing works, referencing relevant papers such as \cite{Zhang2020} and \cite{Liu2020}.
\end{todolist}

\printglossary
\printglossary[type=\acronymtype]



\printbibliography

\end{document}
